
%%%%%%%%%%%%%%%%%%%%%%%%%%%%%%%%%%%%%%%%%%%%%%%%%%%%%%%%%%
%              Author: Emanuele Donno                    %
%                                                        %
%                                                        %
%  Template creato ispirandosi alle linee guida fornite  %
%  per la creazione del proprio lavoro di tesi.          %
%                                                        %
%  N.B.: Non è un documento ufficiale! Riferirsi al      %
%  sito dell'università per scaricare il template        %
%  ufficiale in Word. Questa è solo una mia creazione    %
%  per gli studenti che vogliano cimentarsi nella        %
%  scrittura della propria tesi in LaTeX                 %
%                                                        %
%  Per il logo da utilizzare nel template riferirsi      %
%  al sito dell'università e al template in Word.        %
%                                                        %
%                 Ad astra per aspera!                   %
%                                                        %
%%%%%%%%%%%%%%%%%%%%%%%%%%%%%%%%%%%%%%%%%%%%%%%%%%%%%%%%%%

\documentclass[a4paper, 12pt, notitlepage]{report}
\usepackage[T1]{fontenc}
\usepackage{textcomp}

\usepackage[utf8]{inputenc}
\usepackage[italian]{babel}
\usepackage{verbatim}
\usepackage{listings}

\usepackage{hyperref}
\usepackage{etoc}

\usepackage{float}
\usepackage{amsmath}
\usepackage{amsfonts}
\usepackage{mathtools}

\usepackage{fancyvrb}
\usepackage{xcolor}

\usepackage{graphicx} % graphics package
\graphicspath{ {./media/} } % images folder path

\usepackage{geometry}
\geometry{
    a4paper,
    top=30mm,
    bottom =30mm,
    left=35mm,
    right=35mm
    }


\renewcommand*\contentsname{Indice}

% \unisalento command
\newcommand{\unisalento}{\href{http://www.unisalento.it}{\textbf{\emph{Università del Salento}}}}

% colors for code snippets
\definecolor{codegreen}{rgb}{0,0.7,0.5}
\definecolor{codegray}{rgb}{0.5,0.5,0.5}
\definecolor{codeblue}{rgb}{0,0.5,0.82}
\definecolor{backcolour}{rgb}{0.95,0.95,0.95}

\lstdefinestyle{mystyle}{
    backgroundcolor=\color{backcolour},   
    commentstyle=\color{codegreen},
    keywordstyle=\color{orange},
    numberstyle=\tiny\color{codegray},
    stringstyle=\color{codeblue},
    basicstyle=\ttfamily\footnotesize,
    breakatwhitespace=false,         
    breaklines=true,                 
    captionpos=b,                    
    keepspaces=true,                 
    numbers=left,                    
    numbersep=5pt,                  
    showspaces=false,                
    showstringspaces=false,
    showtabs=false,                  
    tabsize=2
}

\lstset{style=mystyle}

% frontispiece created with \title command
\title{
    \includegraphics[scale=0.3]{unilogo.png}
    
    \vspace{1cm}
    {\Huge{\textbf{Università Del Salento}}}
    
    \normalfont{Corso di Laurea (Magistrale/Triennale) in
    
    (Nome CdL)}
    \par\noindent\rule{\textwidth}{0.4pt}
    
    \vspace{0.3cm}
    
    \par \large{Tesi di Laurea in \emph{\ldots}}
    
    \vspace{0.3cm}

    \par \Large{\textbf{Titolo tesi}}
    
    \vspace{6.0cm}

    \begin{minipage}{\textwidth}
        \par \small{Relatore/Relatrice} 
        \par \emph{Chiar.mo/ma Prof.re/ssa (Nome Cognome)}
        \ \newline
        \par \small{Correlatore}
        \par \emph{Dott. re/ssa (Nome Cognome)}
    \end{minipage}

    \par \hfill \small{Laureando 
    \par \hfill \emph{(Nome Cognome)}
    \par \hfill \emph{Matricola n° \ldots}}
    \par\noindent\rule{\textwidth}{0.4pt}
    \vspace{0.3cm}
    
    \small{ANNO ACCADEMICO 2021/2022}
    \thispagestyle{empty}
    }
    
\author{} % lasciare vuoto
\date{}   % lasciare vuoto

\hypersetup{hidelinks}

\begin{document}
    % frontispiece creation
    \maketitle

    % blank page
    \mbox{}
    \thispagestyle{empty}
    
    % index creation
    \hypersetup{hidelinks}
    \tableofcontents
    \thispagestyle{empty}

    % Chapter: "Introduction"
    \hypersetup{hidelinks}
    \include{./capitoli/introduzione.tex}


    % Chapter 1
    \hypersetup{hidelinks}
    \include{./capitoli/capitolo_1.tex}
    
    % Chapter 2
    \hypersetup{hidelinks}
    \include{./capitoli/capitolo_2.tex}
    
    % Chapter 3
    \hypersetup{hidelinks}
    \include{./capitoli/capitolo_3.tex}
    
    % Chapters: "Conclusions"
    \hypersetup{hidelinks}
    \chapter{Conclusioni e Sviluppi futuri}

    
    % Chapter: "Credits"
    \hypersetup{hidelinks}
    \include{./capitoli/ringraziamenti.tex}

    % List of figures
    \newpage
    \listoffigures
    
    % List of tables
    \newpage
    \listoftables
    
    % Bibliography - bibliography example taken on the web
    \begin{thebibliography}{100} % 100 is a random guess of the total number of %references
        \bibitem{Ref_1} Boney, L., Tewfik, A.H., and Hamdy, K.N., ``Digital Watermarks for Audio Signals," \emph{Proceedings of the Third IEEE International Conference on Multimedia}, pp. 473-480, June 1996.
        \bibitem{Ref_2} Goossens, M., Mittelbach, F., Samarin, \emph{A LaTeX Companion}, Addison-Wesley, Reading, MA, 1994.
        \bibitem{Ref_3} Kopka, H., Daly P.W., \emph{A Guide to LaTeX}, Addison-Wesley, Reading, MA, 1999.
        \bibitem{Ref_4} Pan, D., ``A Tutorial on MPEG/Audio Compression," \emph{IEEE Multimedia}, Vol.2, pp.60-74, Summer 1998.
    \end{thebibliography}
%%%%%%%%%%%%% end %%%%%%%%%%%%%%%%%%%%%%%%%%%%%%%

\end{document}

